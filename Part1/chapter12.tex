\chapter{反射,元数据和源生成器}
\underline{\makebox[10cm][l]{这章中有什么?}}

\begin{itemize}
    \item 使用自定义特性
    \item 使用反射在运行时检查元数据
    \item 使用dynamic类型
    \item 用ExpandoObject创建动态对象
    \item 用源生成器编译代码
\end{itemize}
\medskip
\underline{\makebox[10cm][l]{本章的代码下载}}

本章的源代码可在本书位于 \url{http://www.wiley.com} 的页面上获取。点击下载链接即可。
你也可以在 \url{https://github.com/ProfessionalCSharp/ProfessionalCSharp2021} 的 1\_CS/ReflectionAndSourceGenerators 目录下找到这些代码。
\\ \\
本章的代码分为以下几个主要示例:
\begin{itemize}
    \item LookupWhatsNew
    \item TypeView
    \item VectorClass
    \item WhatsNewAttributes
    \item Dynamic
    \item DynamicFileReader
    \item CodeGenerationSample
\end{itemize}
所有的样例项目都启用了可空引用类型

\section{运行时代码检查和动态编程}
这章聚焦于自定义特性,反射,动态编程和借助C\#9源生成器在构建过程中的源代码生成。自定义特性是一些机制他允许你去联系自定义元数据和程序元素。这些元数据在编译的时候被创建,并且被嵌入到一个程序集。反射是一个通用术语,它描述了在运行时检查和操控程序元素这种能力。比如,反射允许你去做以下事情:
\begin{itemize}
    \item 枚举一个类型中的成员
    \item 实例化一个新对象
    \item 执行一个对象中的成员
    \item 找出一个类型的信息
    \item 找出一个程序集的信息
    \item 检查被应用到一个类型的自定义特性
    \item 创建并编译一个新程序集
\end{itemize}
这个列表代表了很多的功能并且包含一些被.NET提供的最强大且最复杂的能力。因为一章中没有包含所有反射能力的空间,所以我聚焦于哪些你很可能去用的最频繁的元素。
\\ \\
为了去展示自定义特性和反射,在这章中,你将首先开发一个例子,基于一个频繁发布他的软件升级并且想要获得关于这些被自动记录的升级细节的公司。在这个例子中,你将定义指明程序元素被最后一次修改时的日期和被修改了什么的自定义特性。然后你使用反射开发一个程序,可以在一个程序集中寻找这些特性并且自动显示从给定日期开始的对软件升级的所有细节。
\\ \\
在这章中的另一个例子考虑了一个应用程序,它可以读出或写入一个数据库,并使用自定义特性标记哪些类或属性对应于哪些表或列。通过在运行时从程序集中读取这些特性,这个程序可以自动检索或写入数据到数据库中的恰当位置,而无需对于每个表或列编写特定逻辑。
\\ \\
这一章的第二个方面就是动态编程,自从dynamic类型被添加到C\#4,动态编程就成为C\#语言的一部分。虽然C\#是一个静态类型语言,但动态编程的加入给了C\#语言从C\#内部调用脚本函数的能力。
\\ \\
在本章中,你将了解 dynamic 类型及其使用规则。你还将看到 DynamicObject 的实现是什么样并且如何使用它。DynamicObject 的实现之一 ExpandoObject 也将被涵盖。
\\ \\
本章的第三个大方面是一个C\#9的增强——源生成器。使用源生成器,代码可以在你开始构建流程是被生成。你写的源代码可以被增强,同时还可以利用其他数据源来生成C\#源代码。在本章中,你将看到源生成器检查特性,从而在编译时生成代码,而无需在运行时依赖反射。

\section{自定义特性}
你已经在这本书中看到如何在你的程序中定义多个特性。这些特性已经被Microsoft作为.NET的一部分被定义,并且许多特性都受到了C\#编译器的特殊支持。这意味着对于那行特定的特性,编译器可以按特定方式自定义编译过程——比如,根据 StructLayout 特性中的细节在内存中给结构体布局。
\\ \\
.NET还允许你去定义特性。默认情况下自定义不会对编译过程有任何影响,因为编译器本身并不会识别它们(一会你将看到源生成器,在那儿自定义特性对编译过程有影响)。当这些特性应用于程序元素时,它们会被当作元数据输出到编译后的程序集中。
\\ \\
单独来看,这些元数据可能对文档编写有帮助,但让特性变得真正强大的是:通过使用反射,你的代码可以在运行时读取这些元数据并做出相应的决策。这意味着你定义的自定义特性可以直接影响你代码的运行方式。例如,对于自定义权限类,自定义特性可以被用于去允许声明式代码访问安全检查,自定义特性还可以去关联信息和程序元素,以便之后被测试工具使用,或者在开发允许加载插件或模块的可扩展框架时使用。

\section{编写自定义特性}
为了理解如何编写自定义特性,了解当编译器遇到一个你代码中的元素,并且这个元素带有一个被实现了的自定义特性时做了什么是有用的。假定你有一个C\#属性定义如下:
\begin{verbatim}
    [FieldName("SocialSecurityNumber")]
        public string SocialSecurityNumber
        {
            get {
                // ...
\end{verbatim}
编译器期望找到一个带有这个名字的类,并且期望这个类直接或间接继承自 System.Attribute。编译器还期望这个类包括了控制特性使用的信息。特别的,这个特性类需要具体说明一下内容:
\begin{itemize}
    \item 可以应用该特性的程序元素的类型(类、结构体、属性、方法等)
    \item 将特性多次应用于同一程序元素是否合法
    \item 当特性应用于类或接口时,是否由派生类和子接口继承
    \item 特性用到的强制和可选参数
\end{itemize}
如果编译器不能找到一个相应的特性类或找到了但是你使用特性的方式不符合特性类的信息,编译器将引发编译错误。
\\ \\
继续这个例子,假定你已经定义了 FieldName 特性如下:
\begin{verbatim}
    [AttributeUsage(AttributeTargets.Property,
        AllowMultiple=false, Inherited=false)]
    public class FieldNameAttribute: Attribute
    {
        private string _name;
        public FieldNameAttribute(string name) => _name = name;
    }
\end{verbatim}
下一节将讨论这个定义中的每一个元素

\subsection{详述 AttributeUsage 特性}
首先要注意的是特性类被标记有一个特性——System.AttributeUsage 特性。这是一个被微软定义的特性,用来给编译器提供特殊支持。AttributeUsage 的首要目的是去识别可以应用你自定义特性的程序元素的类型。这个信息被 AttributeUsage 的第一个参数提供。这个参数是强制性的,并且是可枚举类型:AttributeTargets 。在之前的例子中,你已经指示了 FieldName 特性仅仅可以被应用与属性,这是好的,因为这正是你在前面的代码片段中应用它的地方。AttributeTargets 枚举类型定义了一些成员去应用特性到程序集,类,构造函数,字段,事件,方法,接口,结构体,返回值等等。
\\ \\
注意,当应用一个特性到你的程序元素时,你直接将特性放到元素前的方括号中。然而,之前列表中的两个值和任何程序元素都没有关联:Assembly 和 Module。特性可以作为一个整体应用于程序集或模块,而不是你代码中的元素;在这个情况下,特性可以被放到你源代码的任何地方,但是他必须有 assembly 或 module 前缀:
\begin{verbatim}
    [assembly:SomeAssemblyAttribute(Parameters)]
    [module:SomeAssemblyAttribute(Parameters)]
\end{verbatim}
当指定有效的目标元素给自定义特性时,你可以用位运算符 OR 组合这些值。例如,如果你想去指示你的 FieldName 特性可以被应用到属性和字段上,你将使用下面代码:
\begin{verbatim}
    [AttributeUsage(AttributeTargets.Property | AttributeTargets.Field,
        AllowMultiple=false, Inherited=false)]
    public class FieldNameAttribute: Attribute
\end{verbatim}
你也可以用 AttributeTargets.All 去表明你的特性可以被应用到所有程序元素类型。AttributeUsage 特性还包含了两个其他参数:AllowMultiple 和 Inherited。它们被规定使用 <ParameterName>=<ParameterValue> 语法而不是仅仅指定这些参数的值。这些参数是可选的——你可以省略他们。
\\ \\
AllowMultiple 参数表明了是否一个特性可以被多于一次的应用与同一项。当它被设为 false 时,编译器应该引发错误,当看到类似这样的情况:
\begin{verbatim}
    [FieldName("SocialSecurityNumber")]
    [FieldName("NationalInsuranceNumber")]
    public string SocialSecurityNumber
    {
        // ...
\end{verbatim}
如果 Inherited 参数被设为 true,应用于类或接口的特性会自动应用于所有派生类或子接口。如果特性被应用与一个方法或属性,他将会自动应用与任何对于那个方法或属性的重写(override),以此类推。

\subsection{详述特性参数}