\documentclass[11pt,twoside,a4paper]{ctexbook}
\setcounter{tocdepth}{1}  % 1对应section级别的目录

\usepackage{hyperref} % 书签
\usepackage{parskip} % 段落间垂直距离
\usepackage{enumitem} % 用于自定义列表环境
\usepackage{etoolbox} % 对环境进行钩子操作(verbatim)

\setlength{\parindent}{0pt} % 缩进为0
\setlength{\parskip}{1em} % 设置段落之间的间距为1em

% 列表无额外间距
\setlist[itemize]{nosep}
% verbatim 环境无段落间距
\AtBeginEnvironment{verbatim}{\setlength{\parskip}{0pt}}

\begin{document}

% \tableofcontents % 目录

\frontmatter        % 切换为罗马数字页码 i, ii, iii...

\chapter*{引言}
\textbf{尽管} .NET 在 2000 年被发布,但它没有变成一个过时的技术。相反,.NET 对开发者的吸引力持续增长,自从它开源并且不仅可以在 Windows 使用还能在 Linux 上使用。.NET 还可以在客户端的浏览器中运行——不需要下载扩展——通过使用 WebAssembly 标准。

随着对 C\# 和 .NET 的新增强功能不断推出,其关注点不仅在于性能提升,还在于易用性。.NET 越来越成为新开发者的选择。

C\# 对长期开发者同样有吸引力。每年,Stack Overflow 询问开发者他们的最爱,最害怕和最想要的变成语言和框架。C\# 已经持续几年在“最爱的语言”排列前 10。ASP.NET Core 现在是“最爱的web框架”首位。.NET Core 在“最爱的其他框架/库/工具”类别中第一。详见 \url{https://insights.stackoverflow.com/survey/2020}

当你使用 C\# 和 ASP.NET Core,你可以创建 web 应用程序和服务(包括微服务),运行在 Windows,Linux 和 Mac 上。你可以用 Windows 运行时去创建本地 Windows 程序,使用 C\#,XAML 和 .NET。你可以创建类库,它可以在 ASP.Net Core,Windows 程序和 .Net MAUI 中共享。

大多数本书中的例子是构建和运行在 Windows 或 Linux 系统上的。例外情况是仅在 Windows 平台上运行的 Windows 应用程序示例。你可以使用 Visual Studio,Visual Studio Code,或 Visual Studio for the Mac 作为开发环境;只有 Windows 程序示例需要 Visual Studio。

\section*{.NET 的世界}
.NET 有长的历史;第一个版本被发布于 2002 年。完全重写的新一代的 .NET(.NET Core 1.0 在 2016 年)是非常年轻的。最近,很多来自旧 .NET 版本的功能已经被带到了 .NET Core 去简化迁移体验。

当创建新的应用程序,没有理由不迁移到新的 .NET 版本。旧的程序是否应该呆在旧版本的 .NET 还是迁移到新版本取决于使用的功能,迁移的难度和迁移后你能获得什么优势。这个问题最好的选择需要通过逐个应用程序的分析来考虑。

新 .NET 提供了容易的方式去创建 Windows 和 web 程序和服务。你可以创建微服务在 Kubernetes 集群中的 Docker 容器中运行;创建 web 程序返回 HTML,JavaScript 和 CSS;创建 web 程序返回 HTML,JavaScript 和 .NET 二进制文件,他使用 WebAssembly 以安全且标准的方式运行在客户端浏览器里。你可以以传统的方式创建 Windows 程序使用 WPF 和 Windows Forms 并利用现代的 XAML 功能和控件,它们支持用 WinUI 实现流畅的设计,以及通过.NET MAUI 开发移动应用程序。

.NET 使用现代模式。依赖注入被构建进核心服务,比如 ASP.NET Core 和 EF Core,这不仅使单元测试更容易,而且允许开发人员容易的增强和更改这些技术的功能。

.NET 能运行在多平台上。除了 Windows 和 macOS,许多 Linux 环境也被支持,比如 Alpine,CentOS,Debian,Febora,openSUSE,Red Hat,SLES 和 Ubuntu。

.NET 是开源的(\url{https://github.com/dotnet})并且免费获得。你可以找到 C\# 编译器的会议记录(\url{https://github.com/dotnet/csharplang}),C\# 编译器的源代码(\url{https://github.com/dotnet/Roslyn}),.NET 运行时和类库的源代码(\url{https://github.com/dotnet/runtime}),和 ASP.NET Core的源代码(\url{https://github.com/dotnet/aspnetcore}),包括 Razor Pages,Blazor 和 SignalR。

这里是一些对于新 .NET 功能特点的总结:
\begin{itemize}
    \item .NET 是开源的。
    \item .NET 使用现代模式。
    \item .NET 支持在多平台上开发
    \item ASP.NET Core 可以允许在 Windows 和 Linux 上。
\end{itemize}

\section*{C\# 的世界}
当 C\# 在 2002 年被发布,他是一种为了 .NET Framework 开发的语言。C\# 的设计思想来自 C++,Java 和 Pascal。Anders Hejlsberg 从 Borland 来到微软并带来了Delphi语言开发的经验。在 Microsoft,Hejlsberg 工作于 Microsoft 版的 Java,叫做 J++,在创造 C\# 前。

\begin{framed}
\textbf{注意} 今天,Anders Hejlsberg 已经转移工作到 TypeScript( 虽然他依然影响 C\#),并且 Mads Torgersen 是 C\# 的项目领头人。C\# 的改进在 \url{https://github.com/dotnet/csharplang} 上公开讨论,而且你可以阅读 C\# 语言的建议和会议记录。你也可以提交自己的 C\# 建议。
\end{framed}

C\# 最初不仅是一门面向对象的通用编程语言,更是一种基于组件的编程语言,支持属性、事件、特性(注解)和构建程序集(包含元数据的二进制文件)。

随着时间的推移,C\# 被增强,通过加入泛型、语言集成查询(LINQ)、lambda 表达式、动态特性以及更简便的异步编程。C\# 不是一个简单的编程语言,因为它提供了很多特功能性,但它也在持续发展,不断加入实用的新功能。由此,C\# 不仅仅是一门面向对象或基于组件的语言;他还包含了函数式编程的理念——这些对于开发各类应用的通用语言来说都具有实用价值。

如今,每年都会发布一个新版本的 C\#,C\# 8 加入了 可空引用类型,C\# 9加入了 records 等等,C\# 10 和 .NET 6 在 2021 年发布,C\# 11 和 .NET 7 在 2022 年发布。因为如今变更频繁,查看本书的 GitHub 仓库(详见“源代码”一节)以获取持续更新。

\section*{C\# 的新东西}
每年,一个新版本的 C\# 被发布,伴随着许多新功能特性可以获得。最新的版本包括很多功能特性:可空引用类型,去减少 NullableReferenceException 异常,作为替换,让编译器提供更多帮助;提高生成效率的功能,如 indices 和 ranges;switch 表达式,它可以让 switch 语句显得过时;using 声明等简化语法;模式匹配的提升。顶级语句允许使小型应用程序减少源代码的行数,record——一种 class,可让编译器自动生成用于相等性比较、解构和 with 表达式的模板代码。代码生成器允许自动创建代码当编译器允许。所有这些新特性都在本书中涵盖。

\section*{ASP.NET CORE 的新东西}
ASP.NET Core 现在包含新技术用于创建 web 应用:Blazor Server 和 Blazor WebAssembly。使用 Blazor,你有一个用 C\# 代码编写客户端和服务端的全栈选择。使用 Blazor Server,你创建的包含 HTML 和 C\# 代码的 Razor 组件会在服务器上运行。使用 Blazor WebAssembly,用 C\# 和 HTML 编写的 Razor 组件会在客户端运行——借助 HTML5 标准的 WebAssembly 技术,该技术允许在浏览器中运行二进制代码,且为所有现代 Web 浏览器所支持。

对于创建服务,你现在可以使用 ASP.NET Core 的 gRPC,实现服务间的二进制通信。如果需要传输大量数据,这是服务到服务通信的绝佳选择,以减少带宽需要,同时降低 CPU 和内存占用。

\section*{Windows 的新东西}
对于给 Windows 开发程序,一个新技术结合了通用 Windows 平台和桌面应用的功能:WinUI 3。WinUI 是 Windows 10 应用的原生 UI 平台。使用 WinUI3,你可以用包含编译绑定的现代的 XAML 代码去创建桌面应用。新的控件和 Microsoft 的流畅设计系统是可获得的。这些控件不会像以前的通用 Windows 平台(UWP)那样在 Windows Runtime 上被提供,这些控件独立于 Windows 10 版本开发的,允许您在 Windows 10 1809 及以上版本中使用最新控件。正如 WinUI 提供的路线图所示,这些新控件也将可用于 WPF 应用程序。

\section*{你需要用什么编写和运行 C\# 代码}
.NET 运行在 Windows,Linux 和 Mac 操作系统上。你可以创建和构建你的程序在任何浙西平台上使用 Visual Studio Code(\url{https://code.visualstudio.com})。你可以构建并运行大部分例子在 Windows 或 Linux 上,并使用你选择的 .NET 开发工具。只有 WinUI 程序需要你去使用 Windows 平台,并且,Visual Studio 是最好的选择。构建并运行 WinUI 程序需要的最小版本是 16.10。

当使用 .NET CLI 和 Azure CLI时,命令行承担了一个重要的部分;你可以使用新的 Windows Terminal。在最新的 Windows 10 中,这个终端成为 Windows 的一部分被提供。在老版本上,你可以通过 Microsoft Store 下载它。

大部分 .NET 开发者使用 Windows 平台作为开发机器。当使用 Windows 对于 Linux 的子系统(WSL 2)时,你可以构建并运行你的 .NET 程序在 Linux 环境,你还可以从你的 Windows 环境下载不同的 Linux 发行版,并访问相同的文件。Visual Studio 甚至允许调试你的 .NET 程序,当它们运行在 WSL 2 上 Linux 环境时。

在本书的一些示例中,Microsoft Azure 被展示为一个可选的托管环境,可用于运行 web 应用程序、使用 Azure Functions,以及使用 Entity Framework Core 去访问 SQL Server 和 Azure Cosmos DB。
为此,你可以使用 Microsoft Azure 提供的免费试用服务;访问 \url{https://azure.microsoft.com/free} 进行注册。

\section*{本书涵盖了什么}
本书包括了一下四个主要部分:
\begin{itemize}
    \item C\# 语言
    \item 使用 .NET 中基础的类库
    \item 开发 web 应用和服务
    \item 开发 Windows 应用
\end{itemize}
让我们进入不同的部分和所有章节了解详情。

\section*{第一部分,“C\# 语言”}
本书的第一部分涵盖了 C\# 编程语言的所有方面。你将学习语法选项,并了解 C\# 语法如何从 .NET 中集成类和接口。这部分提供了一个 C\# 语言的好基础。本节不假定您了解任何特定的编程语言,但是它假设你是一位经验丰富的程序员。在深入了解 C\# 的高级功能之前,您首先要了解 C\# 的基本语法和数据类型。
\begin{itemize}
    \item 第一章,“.NET 应该程序和工具”,包括了你创建 .NET 程序需要的知识。你学习 .NET CLI 并创建一个 Hello World 程序,使用 C\# 9 的顶级语句。
    \item 第二章,“C\# 核心”,深入了解 C\# 的核心特性,给你关于顶级语句的细节和信息,在声明变量和数据类型方面。
    \item 第三章,“Classes、Records、Structs 和 Tuples”,给你创建引用或值类型的信息,创建和使用 tuples,并使用 C\# 9 对创建和使用 records 的增强。
    \item 第四章,“用 C\# 面向对象编程”,深入 C\# 的面向对象技术细节,演示用于面向对象的所有的 C\# 关键字。
    \item 第五章,“运算符与类型转换”,解释了 C\# 运算符,你还将学习如何为自定义类型重载标准运算符。
    \item 第六章,“数组”,不会止步于简单的数组;你将学习使用多维数组和交错数组,使用 Span 类型去访问数组,以及使用新的 index 和 range 运算符去访问数组。
    \item 第七章,“委托,Lambda 表达式和事件”,包括了 .NET 对方法的指针,带有闭包的 lambda 表达式和.NET 事件。
    \item 第八章,“集合”,深入不同类型的集合,比如列表、队列、栈、字典和不可变集合。本章还提供了决定在什么情况下使用哪个集合所需的信息。
    \item 第九章,“语言集成查询”,介绍了 C\# 语言集成的查询功能,可以从你的集合中查询数据。你还将学到如何在查询中使用多核 CPU,还有表达式树背后用的是什么,当你使用 LINQ 通过 Entity Framework Core 去访问你的数据库时。
    \item 第十章,“错误和异常”,包括了你应该如果解决错误,抛出并捕获异常,和当捕获它们时过滤异常。
    \item 第十一章,“任务和异步编程”,展示了 C\# 关键字 async 和 await 的实际应用——不仅适用于基于任务的异步模式,还适用于异步流,这是 C\# 8 中的新特性。
    \item 第十二章,“反射、元数据和源生成器”,
    \item 第十三章,“托管与非托管内存”,这是第一部分的最后一章,它不仅展示了使用 IDisposable 接口伴随着 using 语句和新的 using 声明,还展示了使用 Span 类型与托管和非托管内存结合使用。你还可以了解到如何在 Windows 和 Linux 环境中使用 Platform Invoke。
\end{itemize}

\mainmatter          % 开始正文,页码切换为阿拉伯数字

\part{C\#语言}
% \begin{itemize}
%     \item 第一章:.NET 应用程序和工具
%     \item 第二章:核心 C\#
%     \item 第三章:类,记录,结构体和元组
%     \item 第四章:用 C\# 面向对象编程
%     \item 第五章:运算符与类型转换
%     \item 第六章:数组
%     \item 第七章:委托,Lambda 表达式和事件
%     \item 第八章:集合
%     \item 第九章:语言集成查询(LINQ)
%     \item 第十章:错误和异常
%     \item 第十一章:任务和异步编程
%     \item 第十二章:反射,元数据和源生成器
%     \item 第十三章:托管与非托管内存
% \end{itemize}
% \chapter{.NET 应用程序和工具}
% \chapter{核心 C\#}
% \chapter{类,记录,结构体和元组}
% \chapter{用 C\# 面向对象编程}
% \chapter{运算符与类型转换}
% \chapter{数组}
% \chapter{委托,Lambda 表达式和事件}
% \chapter{集合}
% \chapter{语言集成查询(LINQ)}
% \chapter{错误和异常}
% \chapter{任务和异步编程}
\chapter{反射,元数据和源生成器}
\underline{\makebox[10cm][l]{这章中有什么?}}

\begin{itemize}
\item 使用自定义特性
\item 使用反射在运行时检查元数据
\item 使用dynamic类型
\item 用ExpandoObject创建动态对象
\item 用源生成器编译代码
\end{itemize}
\medskip
\underline{\makebox[10cm][l]{本章的代码下载}}

本章的源代码可在本书位于 \url{http://www.wiley.com} 的页面上获取。点击下载链接即可。
你也可以在 \url{https://github.com/ProfessionalCSharp/ProfessionalCSharp2021} 的 1\_CS/ReflectionAndSourceGenerators 目录下找到这些代码。
\\ \\
本章的代码分为以下几个主要示例:
\begin{itemize}
\item LookupWhatsNew
\item TypeView
\item VectorClass
\item WhatsNewAttributes
\item Dynamic
\item DynamicFileReader
\item CodeGenerationSample
\end{itemize}
所有的样例项目都启用了可空引用类型

\section{运行时代码检查和动态编程}
这章聚焦于自定义特性,反射,动态编程和借助C\#9源生成器在构建过程中的源代码生成。自定义特性是一些机制他允许你去联系自定义元数据和程序元素。这个元数据在编译的时候被创建,并且被嵌入到一个程序集。反射是一个通用术语,它描述了在运行时检查和操控程序元素这种能力。比如,反射允许你去做以下事情:
\begin{itemize}
\item 枚举一个类型中的成员
\item 实例化一个新对象
\item 执行一个对象中的成员
\item 找出一个类型的信息
\item 找出一个程序集的信息
\item 检查被应用到一个类型的自定义特性
\item 创建并编译一个新程序集
\end{itemize}
这个列表代表了很多的功能并且包含一些被.NET提供的最强大且最复杂的能力。因为一章中没有包含所有反射能力的空间,所以我聚焦于哪些你很可能去用的最频繁的元素。
\\ \\
为了去展示自定义特性和反射,在这章中,你将首先开发一个例子,基于一个频繁发布他的软件升级并且想要获得关于这些被自动记录的升级细节的公司。在这个例子中,你将定义指明程序元素被最后一次修改时的日期和被修改了什么的自定义特性。然后你使用反射开发一个程序,可以在一个程序集中寻找这些特性并且自动显示从给定日期开始的对软件升级的所有细节。
\\ \\
在这章中的另一个例子考虑了一个应用程序,它可以读出或写入一个数据库,并使用自定义特性标记哪些类或属性对应于哪些表或列。通过在运行时从程序集中读取这些特性,这个程序可以自动检索或写入数据到数据库中的恰当位置,而无需对于每个表或列编写特定逻辑。
\\ \\
这一章的第二个方面就是动态编程,自从dynamic类型被添加到C\#4,动态编程就成为C\#语言的一部分。虽然C\#是一个静态类型语言,但动态编程的加入给了C\#语言从C\#内部调用脚本函数的能力。
\\ \\
在本章中,你将了解dynamic类型及其使用规则。你还将看到DynamicObject的实现是什么样并且如何使用它。DynamicObject的实现之一ExpandoObject也将被涵盖。
\\ \\
本章的第三个大方面是一个C\#9的增强——源生成器。使用源生成器,代码可以在你开始构建流程是被生成。你写的源代码可以被增强,同时还可以利用其他数据源来生成C\#源代码。在本章中,你将看到源生成器检查特性,从而在编译时生成代码,而无需在运行时依赖反射。

\section{自定义特性}
% \chapter{托管与非托管内存}

\end{document}
