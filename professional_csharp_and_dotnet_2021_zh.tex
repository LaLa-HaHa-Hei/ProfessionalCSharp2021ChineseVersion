\documentclass[11pt,twoside,a4paper]{ctexbook}

\usepackage{hyperref} % 书签
\usepackage{parskip} % 段落间垂直距离
\usepackage{enumitem} % 用于自定义列表环境
\usepackage{etoolbox} % 对环境进行钩子操作(verbatim)

\setlength{\parindent}{0pt} % 缩进为0
\setlength{\parskip}{1em} % 设置段落之间的间距为1em

% 列表无额外间距
\setlist[itemize]{nosep}
% verbatim 环境无段落间距
\AtBeginEnvironment{verbatim}{\setlength{\parskip}{0pt}}

\begin{document}

% \tableofcontents % 目录

\part{C\#语言}
% \begin{itemize}
%     \item 第一章:.NET 应用程序和工具
%     \item 第二章:核心 C\#
%     \item 第三章:类,记录,结构体和元组
%     \item 第四章:用 C\# 面向对象编程
%     \item 第五章:运算符与类型转换
%     \item 第六章:数组
%     \item 第七章:委托,Lambda 表达式和事件
%     \item 第八章:集合
%     \item 第九章:语言集成查询(LINQ)
%     \item 第十章:错误和异常
%     \item 第十一章:任务和异步编程
%     \item 第十二章:反射,元数据和源生成器
%     \item 第十三章:托管与非托管内存
% \end{itemize}
% \chapter{.NET 应用程序和工具}
% \chapter{核心 C\#}
% \chapter{Classes、Records、Structs 和 Tuples}
% \chapter{用 C\# 面向对象编程}
% \chapter{运算符与类型转换}
% \include{Part1/chapter6}
% \chapter{委托,Lambda 表达式和事件}
% \include{Part1/chapter8}
% \include{Part1/chapter9}
% \chapter{错误和异常}
% \chapter{任务和异步编程}
\chapter{反射,元数据和源生成器}

\section{运行时检查代码和动态编程}
这章聚焦于自定义特性,反射,动态编程和借助C\#9源生成器在构建过程中的源代码生成。
% \chapter{托管与非托管内存}

\end{document}
